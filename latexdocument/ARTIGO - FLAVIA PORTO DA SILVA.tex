\documentclass[article, a4paper, 12pt, brazil]{abntex2}

\usepackage[brazil]{babel}
\usepackage[utf8]{inputenc}

\usepackage[style=abnt]{biblatex} % estilo de referência da ABNT

\usepackage[lmargin=3cm,tmargin=3cm,rmargin=2cm, bmargin=2cm]{geometry} % Margins do documento

\addbibresource{biblio.bib} % Referências

\begin{document}
%-------------------------------------------CAPA--------------------------------------
   \begin{center} % capa
      \textbf{
         FACULDADE DOM ALBERTO\\ % NOME DA FACULDADE
      }

      \vspace{7cm}

      \textbf{
         OS DESAFIOS DO MARKETING NA GESTÃO EMPRESARIAL\\ % TITULO DO TRABALHO
      }

      \vspace{5cm}

      \textbf{
         FLAVIA PORTO DA SILVA\\ % NOMES DOS ALUNOS
      }

      \vspace{10cm}

      \textbf{
         2022 \-- SP % ANO E ESTADO
      }
   \end{center}
    %-------------------------------------------RESUMO--------------------------------------
   \newpage
   \begin{center}
      \section*{RESUMO}
   \end{center}

   \newpage
   %-------------------------------------------ABSTRACT--------------------------------------
   \begin{center}
      \section*{ABSTRACT}
   \end{center}
   %-------------------------------------------OBJETIVOS--------------------------------------
   \pagebreak
   \section{OBJETIVOS}

   \subsection{OBJETIVO GERAL}
  % OBJETIVO GERAL
   Criar um panorâma atual da importância de um planejamento de marketing na gestão de empresas brasileiras.

   \subsection{OBJETIVOS ESPECIFICOS}
   % OBJETIVOS ESPECIFICOS

   \begin{itemize}[noitemsep]
      \item {Investigar a situação do marketing na gestão de empresas brasileiras.}\\
      \item {Descobrir o papel do marketing na administração de empresas.}\\
      \item Comparar a linha do tempo do marketing (gerar historico de evoluções).\\
      \item Achar a relação do marketing e a gestão de empresas.\\
      \item Concluir qual a importância do marketing na gestão de empresas.\\
   \end{itemize}
   \pagebreak
   %--------------------------------REVISÃO BIBLIOGRÁFICA--------------------------------------
   \section{LEVANTAMENTO BIBLIOGRÁFICO}
   % LEVANTAMENTO BIBLIOGRAFICO
   \subsection{INTRODUÇÃO A GESTÃO DE MARKETING}
   \par Na gestão que integra o marketing, é onde organiza, se cria os métodos estratégicos, planificação, metas, indicativos,critérios e o seu desenvolvimento para atender melhor p seu cliente.\\
   \par Muitas vezes a gestão de marketing na empresa é vista como uma mera postagem para divulgação do seu produto nas redes sociais, mas na verdade ela integra um elo de funções muito mais complexo do que se pode imaginar para que possa gerar resultados.\\
   Assim a gestão na empresa deve ser processo constante para que os resultados sigam firmes, assim tem que obter o marketing na empresa valorizando o seu ponto final a organização para que a empresa possa conseguir oque almeja.\\
   % ---------------------------------
   \par Alguns dos seus processos e critérios é conhecer e entender o seu público-alvo, considerar tudo oque lhe é imposto diante das suas colocações da sua marca e pensar de forma grande é pensando no seu produto final com muito trabalho e de forma detalhista.\\
   \subsection{O QUE É GESTÃO DE MARKETING?}

   \par A gestão de marketing envolve um estudo mercadológico para atender oque se pede, gerar o lucro e acima de tudo seu propósito crucial é satisfazer a necessidade do seu cliente.\\
   \par Coordenar e atingir os propósitos é o papel da gestão , já o marketing atribui ao estudo dos mercados e seus posicionamentos, buscando se posicionar e entender sua concorrência e de quem fornece sua matéria prima, ou algo semi pronto que corresponde a determinada empresa, de modo controlar o e organizado de maneira contínua e padronizada e de quem possa interessar a empresa.\\
   \par Para realizar todas as atividades da empresa é necessário executar no tempo certo todas e suas funções de r.h e oque envolve o dinheiro da empresa, saber sobre os mercados e encontrar respostas sólidas em cada problema ocorrer.\\
   \par Em todo o seu planejamento é necessário que contenha um investimento que comporte a empresa para que alcance o retorno esperado, as suas ferramentas para a venda como o uso de sites tudo que inclui o virtual para as pessoas conhecerem a empresa é como ela funciona e oque oferece para o seu cliente como o cem que é uma ferramenta que possibilita que o cliente deixe a sua opinião para que a empresa compreenda ainda mais os seus clientes e saiba oque pode se fazer no que está faltando aprimorar na sua empresa, a sua divulgação estratégica para que se resuma em êxito e não é apenas só a postagem, é o seu logo, sua frase, seu slogan,a foto que define a imagem da empresa, sua "jogada de mestre".\\
   \subsection{O QUE É 5W2H?}

   \par A ferramenta se opõe para organizar os dados em planilhas ou tabelas.para as empresas para o seu planejamento, funções, objetivos e onde quer chegar e como será feito para alcançar l, quanto será gasto, cistos e benefícios, o público destinado, o porquê da empresa, qual o fato de estarem no mercado, como é feito, o espaço é seus custos de produção. É assim um trabalho árduo e constante para obter os melhores resultados para entregar para o seu cliente final, ela se opõe para organizar os dados em planilhas ou tabelas.\\
   % ---------------------------------
   \par Desde a criação da empresa o marketing precisa caminhar lado a lado dela para dar todo o suporte necessário, começando onde é oque será  planejado para a produção de uma nova mercadoria, os serviços a serem feitos para o respectivos consumidores e o final a sua entrega e negociação para a venda. Existem também peculiaridades a mais do que a venda, antes mesmo de fazer a venda, existe estudos de viabilização, ou seja se o produto em aí irá gerar lucro para a empresa é irá crescer a venda para aquele setor que se enquadra para que garanta o investimento.\\
   \par É para isso o marketing contribui para que a proposta do produto seja  a melhor possível e garanta a compra do cliente com uma campanha que o cliente saiba que é bom e quando chega supra tufo aquilo da proposta feita antes da sua compra.\\
   \par A gestão da empresa têm que adaptar -se a escolha de um dos componentes do marketing para "ganhar" o seu público.Pensando também nos concorrentes, pois a empresa precisa que o dela seja uma alternativa que não necessite encontrar entre os seus concorrentes, trazendo inovações nas suas embalagens, integrando a mudanças e modernizando e gerando sustentabilidade para contribuir com um bem maior a natureza e todo o seu ecossistema, aprimorando seus métodos buscando a valorização do cliente com diferenciação.\\
   \par Adaptando seu portfólio para melhores perspectivas e escolha do produto para sempre ofertar qualidade para o cliente final.\\
   % ---------------------------------
   \par Gerir uma empresa é uma das melhores e a principal coisa a se fazer desde o seu início, para um empreendedor o fator crucial é entender e compreender todos os benefícios da gestão para realizar o marketing na sua empresa.\\
    \par Nessa questão não é somente ter a confiança do seu produto ou como está o seu produto em mercado.\\
   A gestão da empresa deve se assegurar com políticas internas, funções estrategistas importantes para que a empresa execute todos os seus setores tendo em vista a intensidade de valor aos seu colaboradores para que se conduza a empresa de forma harmoniosa,  contribuindo para que faça acontecer e tenha constante crescimento do negócio.\\
   Outro fator relevante na gestão no mercado, questões de competitividade da concorrência e nas vendas.\\
   \par Os fatores que poder trazer mudança nos caminhos da empresa os moldes ideais são aqueles que se adequam as suas atividades dentro da empresa.\\
   \par Assim considerando a parte interessada de quem é empreendedor e do cliente.
   Os funcionários de uma empresa devem ser cruciais no funcionamento dela na sua condução e base e rendimento e evolução, além dos segmentos de softwares onde acontecem suas reduções no quadro de funcionários.\\
   \par Assim conciliando bons resultados de uma ótima gestão integrada na empresa juntamente com o marketing,  onde se evidenciam o crescimento, produtividade e financeiramente.
   Na gestão de finanças se adequa na sua produção e para a motivação dos profissionais que trabalham no seu devido setor dentro da empresa ecomo vai ser o seu rendimento dos colaboradores.\\
   \par Para sua melhor gestão um dos fatores destacados acima é que fazem um controle de desperdícios de produção,  de gastos desnecessários e  em fornecimentos e  no controle de aquisições.\\
   % ---------------------------------
   \par Destacando um dado muito importante no que a gestão empresarial é um investimento imprescindível no que o empresário ou empreendedor não perde o dinheiro ele garante segurança para seguir firme.\\
   \par Dicas que seguem para aderir uma boa prática é o online, que é o ERP que é um sistema computacional que ajuda a controlar e organizando todos os dados e ajudando o profissional que opera esse sistema de software para que se torna mais ágeis os serviços a serem prestados e tome menos tempo.
   As possibilidades ao aderir o sistema ERP são:\\

    \begin{itemize}[noitemsep]
      \item Financeiro
      \item Entrada e saída de custos.
      \item Diminui o tempo dos processos a serem realizados.
      \item Organização de estoque, evitando altos gastos e prejuízos.
      \item Alta lucratividade
      \item Facilidade
      \item Melhor  gerenciamento.
      \item Orçamentos organizados
      \item Melhor entendimento em questões das vendas e emissão fiscal.
      \item Simplificação das cotações para os clientes\cite{fulano}.
   \end{itemize}


   \pagebreak

   \section{METODOLOGIA}
   %  SUA METODOLOGIA
   \pagebreak

   \section{RESULTADOS}
   \pagebreak
   % SEUS RESULTADOS
   \section{CONCLUSÃO}
   % SUA CONCLUSÃO
   \pagebreak

\printbibliography[title={REFERÊNCIAS}]
\end{document}


